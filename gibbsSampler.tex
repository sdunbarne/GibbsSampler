%%% -*-LaTeX-*-
%%% gibbsSampler.tex.orig
%%% Prettyprinted by texpretty lex version 0.02 [21-May-2001]
%%% on Tue Oct 25 09:38:00 2022
%%% for Steve Dunbar (sdunbar@family-desktop)

\documentclass[12pt]{article}

\input{../../../../etc/macros} %% \input{../../../../etc/mzlatex_macros}
\input{../../../../etc/pdf_macros}

% Define a new command so Boltzmann constant is not confused.

\bibliographystyle{plain}

\begin{document}

\myheader \mytitle

\hr

\sectiontitle{The Gibbs Sampler}

\hr

\usefirefox

\hr

% \visual{Study Tip}{../../../../CommonInformation/Lessons/studytip.png}
% \section*{Study Tip}

% \hr

\visual{Rating}{../../../../CommonInformation/Lessons/rating.png}
\section*{Rating} %one of
% Everyone: contains no mathematics.
% Student: contains scenes of mild algebra or calculus that may require guidance.
Mathematically Mature:  may contain mathematics beyond calculus with
proofs.  % Mathematicians Only: prolonged scenes of intense rigor.

\hr

\visual{Section Starter Question}{../../../../CommonInformation/Lessons/question_mark.png}
\section*{Section Starter Question}

What is Bayes' Rule for \( 2 \) jointly distributed continuous random
variables \( X \) and \( Y \)?

\hr

\visual{Key Concepts}{../../../../CommonInformation/Lessons/keyconcepts.png}
\section*{Key Concepts}

\begin{enumerate}
    \item
        The Gibbs sampler is an algorithm for generating random
        variables from a marginal distribution indirectly, without
        having to calculate the density.  Gibbs sampling is based only
        on elementary properties of Markov chains.
    \item
        The simple case of a \( 2 \times 2 \) table with multinomial
        sampling clearly illustrates the Markov chain nature of the
        process.
    \item
        The simple case of a bivariate normal density with Gibbs
        sampling clearly illustrates Gibbs sampling for continuous
        distributions.
    \item
        A simple Bayesian model for a spam filter illustrates typical
        Gibbs sampling.
    \item
        Suppose \( f(x_1, x_2, \dots, x_N) \) is a probability
        distribution in which the variables represent parameters of a
        statistical model.  Gibbs sampling obtains point and interval
        estimates for these parameters.
    \item
        Suppose \( X \sim N(\mu, 1/\tau) \) with \( \mu \) and \( \tau \)
        unknown.  Based on a reasonably sized sample, Gibbs sampling
        obtains the posterior distributions of \( \mu \) and \( \tau \).
    \item
        Hierarchical Bayesian models naturally describe the connections
        among data, observed parameters and other unobserved parameters.
        A simple three-level hierarchical model uses Bayes' rule to bind
        together data, \( X \), a parameter to be estimated, \( \lambda \),
        and an additional hyper-parameter, \( \beta \).
    \item
        In image degradation, a version of the Metropolis algorithm
        called Gibbs sampling the configuration maximizes \( \Prob{\omega
        \given \omega^{\text{blurred}}} \), called the maximum a
        posteriori estimate.
\end{enumerate}

\hr

\visual{Vocabulary}{../../../../CommonInformation/Lessons/vocabulary.png}
\section*{Vocabulary}
\begin{enumerate}
    \item
        The \defn{Gibbs sampler} is a technique for generating random
        variables from a marginal distribution%
        \index{marginal distribution}
        indirectly, without having to calculate the density.
    \item
        \defn{Gibbs sampling} obtains a Gibbs sequence iteratively by
        alternately generating values from
        \begin{align*}
            X'_j &\sim f(x \given Y'_j = y'_j) \\
            Y'_{j+1} &\sim f(y \given X'_j = x'_j).
        \end{align*}
    \item
        The functional correspondence between the Beta families for the
        prior and the posterior and the Binomials for the likelihood
        makes them a \defn{conjugate pair}.%
        \index{conjugate pair}
    \item
        \defn{Hierarchical Bayesian models} naturally describe the
        connections among data, observed parameters and other unobserved
        parameters, sometimes called \defn{latent variables}. A simple
        three-level hierarchical model uses Bayes' rule to bind together
        data, \( X \), a parameter to be estimated, \( \lambda \), and
        an additional hyper-parameter, \( \beta \).
    \item
        The configuration maximizing the probability of a configuration
        given a randomly changed or blurred configuration, \( \Prob{\omega
        \given \omega^{\text{blurred}}} \), is the \defn{maximum a
        posteriori estimate}.
    \item
        A version of the Metropolis algorithm called \defn{Gibbs
        sampling}.  maximizes \( \Prob{\omega \given \omega^{\text{blurred}}}
        \), called the \defn{maximum a posteriori estimate}.
    \item
        A probability distribution whose conditional probabilities
        depend on only the values in a neighborhood system is called a
        \defn{Gibbs distribution} and is part of a larger notion called
        a \defn{Markov random field}.
\end{enumerate}

\hr

\section*{Notation}
\begin{enumerate}
    \item
        \( f(x, y_1, \dots, y_p) \) -- a joint density
    \item
        \( f_{X}(x) = \int \dots \int f(x, y_1, \dots, y_p) \df{y_1}
        \dots \df{y_p} \) -- a marginal density
    \item
        \( p \) -- number of parameters in the joint density
    \item
        \( X_1, \dots, X_m \sim f(x) \) -- sample from the density \( f(x)
        \)
    \item
        \( m \) -- number of elements in the sample
    \item
        \( (X, Y) \) -- Starting pair of random variables
    \item
        \( f(x \given y) \) and \( f(y \given x) \) -- conditional
        distributions
    \item
        \( Y'_0, X'_0, Y'_1, X'_1, Y'_2, X'_2, \dots, Y'_n, X'_n \) -- a
        \emph{Gibbs sequence} of random variables
    \item
        \( p_1, p_2, p_3, p_4 \) -- probabilities in a \( 2 \times 2 \)
        table
    \item
        \( A_{y \given x} \) and \( A_{x \given y} \) -- conditional
        probability of \( Y \) given \( X \) and of \( X \) given \( Y \)
        expressed in matrices
    \item
        \( A_{x \given x} = A_{ y \given x } A_{x \given y} \) -- the
        transition probability matrix for the \( X' \) sequence
    \item
        \( f_n = (f_n(0), f_n(1)) \) -- the marginal probability
        distribution of \( X'_n \)
    \item
        \( f \) -- stationary distribution satisfying \( f A_{x \given x}
        = f \)
    \item
        \( \rho \) -- correlation in bivariate normal distribution
    \item
        \( [S=1] \), the event that the email is in fact spam
    \item
        \( [R=1] \), the event that a spam email is correctly marked
        spam, a \emph{true positive},
    \item
        \( \psi \) -- probability an email is in fact spam
    \item
        \( \eta \) -- the false positive rate
    \item
        \( \theta \) -- the false negative rate
    \item
        \( \tau \) -- probability that a spam email is correctly marked
        spam
    \item
        \( alpha, \beta \) -- shape parameters for the Beta distribution
    \item
        \( f(x_1, x_2, \dots, x_N) \) -- a joint probability density in
        which the variables represent parameters of a statistical model
    \item
        \( \overrightarrow{x}^0 = (x_1^0, x_2^0, \dots, x_N^0) \), \(
        \overrightarrow{x}^1 = (x_1^1, x_2^1, \dots, x_N^1) \) --
        sequence of samples
    \item
        \( X \sim N(\mu, 1/\tau) \) -- normal distribution with \( \mu \)
        the population mean, and \( \tau \) is the population precision
    \item
        \( \bar{x}, s^2 \)-- sample mean and variance
    \item
        \( T \) -- number of iterations in Gibbs sampling
    \item
        \( X \) -- parameter to estimate
    \item
        \( \lambda, \beta \) -- parameter and hyper-parameter in
        Bayesian hierarchical model
    \item
        \( \alpha, \beta, \gamma, \delta \) parameters for distributions
\end{enumerate}

\visual{Mathematical Ideas}{../../../../CommonInformation/Lessons/mathematicalideas.png}
\section*{Mathematical Ideas}

The \defn{Gibbs sampler}%
\index{Gibbs sampler}
is an algorithm for generating random variables from a marginal
distribution indirectly, without having to calculate the density.  This
subsection illustrates the algorithm by exploring several examples.  In
such cases, Gibbs sampling is based only on elementary properties of
Markov chains.

\subsection*{General Theory of Gibbs Sampling}

Given a joint density \( f(x, y_1, \dots, y_p) \),%
\index{joint density}
the goal is to find characteristics of the marginal density%
\index{marginal density}
\[
    f_{X}(x) = \int \dots \int f(x, y_1, \dots, y_p) \df{y_1} \dots \df{y_p},
\] such as the mean or variance.  Often the integrations are extremely
difficult to do, either analytically or numerically.  In such cases the
Gibbs sampler provides an alternate method for obtaining \( f_X(x) \).

Gibbs sampling%
\index{Gibbs sampling}
effectively generates a sample \( X_1, \dots, X_m \sim f(x) \) without
requiring \( f(x) \).  By simulating a large enough sample, the mean,
variance, or any other characteristic of \( f(x) \) can be calculated to
the desired degree of accuracy.

First consider the two-variable case.  Starting with a pair of random
variables \( (X, Y) \), the Gibbs sampler generates a sample from \( f(x)
\) by sampling instead from the conditional distributions \( f(x \given
y) \) and \( f(y \given x) \).  The conditional distributions are often
already known in statistical models.  This is done by generating a \emph
{Gibbs sequence}%
\index{Gibbs sequence}
of random variables
\begin{equation}
    Y'_0, X'_0, Y'_1, X'_1, Y'_2, X'_2, \dots, Y'_n, X'_n.%
    \label{gibbssampler:eq:gibbsseq}
\end{equation}
The first value \( Y'_0 = y'_0 \) is specified and the rest of the
sequence is obtained iteratively by alternately generating values from
\begin{align*}
    X'_j &\sim f(x \given Y'_j = y'_j) \\
    Y'_{j+1} &\sim f(y \given X'_j = x'_j).  \\
\end{align*}
The generation of the sequence is \defn{Gibbs sampling}.%
\index{Gibbs sampling}
Under reasonable general conditions, the distribution of \( X'_n \)
converges to \( f_X(x) \), the true marginal of \( X \), as \( n \to
\infty \).  Thus for \( n \) large enough, the last observation \( X'_n \)
is effectively a sample point from \( f_X(x) \).

\subsection*{Bivariate Binomial Example}

This section discusses Gibbs sampling in detail for the simplest case of
a \( 2 \times 2 \) table.  Suppose \( X \) and \( Y \) are each
marginally Bernoulli random variables with joint distribution (the
probabilities \( X \) are by rows, with \( Y=0 \) the first row, \( Y=1 \)
the second row)
\[
    \bordermatrix{ & 0 & 1 \cr
    0 & p_1 & p_2 \cr
    1 & p_3 & p_4}
\] or in terms of the joint probability distribution
\[
    \begin{pmatrix}
        f_{X,Y}(0,0) & f_{X,Y}(1,0) \\
        f_{X,Y}(0,1) & f_{X,Y}(1,1)
    \end{pmatrix}
    =
    \begin{pmatrix}
        p_1 & p_2 \\
        p_3 & p_4
    \end{pmatrix}
    .
\] For the distribution, the marginal distribution of \( x \) is given
by
\[
    f_X = (f_X(0), f_X(1)) = (p_1 + p_3, p_2 + p_4),
\] a Bernoulli distribution with success probability \( p_2 + p_4 \).
The conditional probabilities can be expressed in two matrices
\[
    A_{y \given x} =
    \begin{pmatrix}
        \frac{p_1}{p_1 + p_3} & \frac{p_3}{p_1 + p_3} \\
        \frac{p_2}{p_2 + p_4} & \frac{p_4}{p_2 + p_4}
    \end{pmatrix}
\] and
\[
    A_{x \given y} =
    \begin{pmatrix}
        \frac{p_1}{p_1 + p_2} & \frac{p_2}{p_1 + p_2} \\
        \frac{p_3}{p_3 + p_4} & \frac{p_4}{p_3 + p_4}
    \end{pmatrix}
\] where \( A_{y \given x} \) has the conditional probabilities of \( Y \)
given \( X \) and \( A_{x \given y} \) has the conditional probabilities
of \( X \) given \( Y \).

As an example, to generate the marginal distribution of \( X \) uses the
\( X' \) sequence from~\eqref{gibbssampler:eq:gibbsseq}.  From \( X'_0 \)
to \( X'_1 \) goes through \( Y'_0 \), so the iteration sequence is \(
X'_0 \to Y'_1 \to X'_1 \) and \( X'_0 \to X'_1 \) is a two-stage Markov
chain, with transition probability
\[
    \Prob{X'_1 \given X'_0} = \sum\limits_y \Prob{X'_1 \given Y'_1 = y}
    \times \Prob{Y'_1 = y \given X'_0}.
\] For example, to go from \( x = 1 \) to \( x = 0 \) takes the dot
product of the second row of \( A_{y \given x} \) with the first column
of \( A_{x \given y} \).  Generally, \( \Prob{X'_1 \given X'_0} \) is
the right matrix multiplication of \( A_{y \given x} \) by \( A_{x
\given y} \), so
\[
    A_{x \given x} = A_{ y \given x } A_{x \given y}
\] is the transition probability matrix for the \( X' \) sequence.  The
matrix that gives \( \Prob{X'_n = x_n \given X'_0 = x_0} \) is \( (A_{x
\given x})^n \).  Letting \( f_n = (f_n(0), f_n(1)) \) denote the
marginal probability distribution of \( X'_n \) then \( f_n = f_0 A^n_{x
\given x} = f_0 (A^{n-1}_{x \given x}) A_{x \given x} = f_{n-1} A_{x
\given x} \).  By the Fundamental Theorem for Markov Chains, \( f_n \to
f \) as \( n \to \infty \) with stationary distribution \( f \)
satisfying \( f A_{x \given x} = f \).  If the Gibbs sequence converges,
the \( f \) satisfying \( f A_{x \given x} = f \) must be the marginal
distribution of \( X \).  In this small example, it is straightforward
to check that \( f_X = ( p_1 + p_3, p_2 + p_4) \) satisfies \( f_X A_{x
\given x} = f_X \).  So stopping the iteration scheme at a large enough
value of \( n \) gives approximately \( f_X \).  The larger the value of
\( n \), the better the approximation.  No general guidance on choosing
such \( n \) is available.  However, one possible approach is to monitor
density estimates from \( m \) independent Gibbs sequences, and choosing
the first point at which these densities agree to a satisfactory degree.

The algebra for the \( 2 \times 2 \) case works for any \( n \times m \)
joint distribution of \( X \)'s and \( Y \)'s.  Analogously define the \(
n \times n \) transition matrix \( A_{X \given X} \) whose stationary
distribution will be the marginal distribution of \( X \).  If either (or
both) of \( X \) and \( Y \) are continuous, then with suitable
assumptions, all of the theory still goes through, so the Gibbs sampler
still produces a sample from the marginal distribution of \( X \).  The
conditional density of \( X_1 \) given \( X_0 \) is \( f_{x_1 \given x_0}
(x_1 \given x_0) = \int f_{X_1 \given Y_1}(x_1 \given y)f_{Y_1 \given X_0}
(y \given x_0) \df{y} \).  Then, step by step, write the conditional
densities of \( X'_2 \given X'_0 \), \( X'_3 \given X'_0 \), \( X'_4
\given X'_0 \), \dots.  Similar to the \( n \)-step transition matrix \(
(A_{x \given x})^n \), derive an ``continuous transition probability
matrix'' with entries satisfying the relationship
\[
    f_{X'_n \given X'_0}(x \given x_0 ) = \int f_{X'_n \given X'_{n-1}}
    (X \given t) f_{X'_{n-1} \given X'_{0}}(t \given x_0) \df{t},
\] the continuous version of the right matrix multiplication.  As \( n
\to \infty \) it again follows that the stationary distribution is the
marginal density of \( X \), the density to which \( f_{X'_n \given X'_0}
\), converges.

\subsection*{Gibbs Sampling from the Bivariate Normal}

Recall that the probability density function for the bivariate normal
distribution%
\index{bivariate normal distribution}
with, for simplicity, means \( 0 \) and variance \( 1 \) for the
variables and correlation \( \rho \) between the two variables is
\[
    f(x, y) = \frac{1}{2\pi \sqrt{1 - \rho^2}} \EulerE^{- \frac{x^2 - 2
    \rho x y + y^2}{2(1-\rho^2)}}.
\] A standard exercise is to show that the marginal densities%
\index{marginal
  distribution}
are in fact
\[
    f_{x}(x) = \frac{1}{\sqrt{2\pi}} \EulerE^{-x^2/2}
\] and
\[
    f_{y}(y) = \frac{1}{\sqrt{2\pi}} \EulerE^{-y^2/2}.
\] Another standard exercise is to show the conditional probability
densities are
\[
    f_{x \given y}(x) = \frac{1}{\sqrt{2\pi (1-\rho^2)}} \EulerE^{-(x-\rho
    y)^2/ (2(1-\rho^2))}
\] and
\[
    f_{y \given x}(y) = \frac{1}{\sqrt{2\pi (1-\rho^2)}} \EulerE^{-(y-\rho
    x)^2/ (2(1-\rho^2))}.
\] That is, \( X_i \given X_j \sim N(x_j, (1-\rho^2)) \).  Figure~%
\ref{fig:gibbssampler:bivarnorm} illustrates the bivariate normal
density with \( \rho = 0.8 \).  The scaling of the \( z \)-axis is
approximately equal to the \( x \) and \( y \) axes, so that the density
appears less ``hill-like'' than is often illustrated.  The marginal
densities are projected onto coordinate planes at \( x = 2 \) and \( y =
2 \).

\begin{figure}
    \centering
    \includegraphics{bvn}
%     \begin{comment}

%         Note that there is a bug in the gs exporter that makes using the
%         graph3 axes not work. See
%         https://sourceforge.net/p/asymptote/discussion/409349/thread/a52414e613/
%         After the bug gets fixed, the following should work.
% \begin{asy}
% settings.outformat = "pdf";
% settings.prc=false;
% settings.render=0;

% import graph3;
% import contour;
% import grid3;
% import palette;

% size(5inches);

% real myfontsize = 12;
% real mylineskip = 1.2*myfontsize;
% pen mypen = fontsize(myfontsize, mylineskip);
% defaultpen(mypen);

% currentprojection=orthographic(-4,-4,2);
% limits((-2,-2,0),(2,2,0.5));

% real rho = 0.8;

% real id( real t) { return t;}
% real n( real t) { return  (1/(sqrt(2 * pi))) * exp( -t^2 / 2); }
% real two( real t) {return 2;}

% path3 margx = graph(id, two, n, -2, 2, operator ..);
% draw( margx, red);
% path3 margy = graph(two, id, n, -2, 2, operator ..);
% draw( margy, red);

% real f(pair z) {
%     real rho = 0.8;
%   return (1/(2 * pi * sqrt(1 - rho^2))) *
%     exp( -(z.x^2 - 2 * rho * z.x * z.y +  z.y^2) / 2);
% }

% surface s=surface(f,(-2,-2),(2,2),20,Spline);

% pen[] Palette=Grayscale();
% draw(s,surfacepen=mean(palette(s.map(zpart),Palette))
%       ,meshpen=black,nolight);

% grid3(new grid3routines [] {XYXgrid, ZXZgrid(2), ZYZgrid(2)},
%       Step=0.25,
%       pGrid=new pen[] {red, blue, black},
%       pgrid=new pen[] {0.5red, lightgray, lightgray});

% xaxis3(  Label("$x$",position=MidPoint,align=SE),
%  Bounds(Min,Min),
%  OutTicks());
%  yaxis3( Label("$y$",position=MidPoint,align=SW),
%  Bounds(Min,Min),
%  OutTicks());
% zaxis3( Label("$z$",position=EndPoint, align=Z),
%   Bounds(Min,Max),
%   InTicks(beginlabel=false,endlabel=true,Label(align=Y)),
%   arrow=Arrow3
%   );
% zaxis3( Label("$z$",position=EndPoint, align=Z),
%   Bounds(Max,Min),
%   InTicks(beginlabel=false,endlabel=true,Label(align=X)),
%   arrow=Arrow3
%   );
% \end{asy}
%    \end{comment}
    \caption{Bivariate normal probability density with marginals.}%
    \label{fig:gibbssampler:bivarnorm}
\end{figure}

Because the marginals are known, using the Gibbs sampler is not
necessary to simulate them.  However, Gibbs sampling using this
elementary example is illustrative as in Figure~%
\ref{fig:gibbsampler:bivarsampler}.  Apply the algorithm for \( 1000 \)
steps.  To allow for convergence to the stationary distribution, the
first \( 500 \) steps are discarded, using only the last \( 500 \)
steps.  The first two subgraphs show the frequencies of the sampled
marginal distributions, along with the theoretical densities in red. The
correspondence is appears close.  Better than a visual comparison of the
densities is to use a Q-Q plot comparing the sample quantiles to the
quantiles of the standard normal distribution.  The straight line
correspondence of the quantiles over several standard deviations
demonstrates the excellent match of simulation to the theoretical
density.  The third row shows the scatter of the bivariate samples, with
the characteristic elliptical distribution The second figure in the
third row shows the autocorrelation of the sample points by connecting
successive points with segments.  The last two plots illustrate the
``white noise''-like aspect of the marginal distributions, as expected.
Altogether, these reinforce the intuition that the Gibbs sample has
produced a satisfactory sample of the marginal distributions.

\begin{figure}
    \centering
    \includegraphics[scale=0.75]{bivarsampler}
    \caption{Outcomes from Gibbs sampling of a bivariate normal
    distribution, illustrating the quality of the simulation.}%
    \label{fig:gibbsampler:bivarsampler}
\end{figure}

\subsection*{A Gibbs Sampler for a Spam Filter}

Consider a Bayesian Inference model for an email spam filter.%\index{spam filter} The goal
is to estimate the prevalence \( \psi \) of email spam without having to
directly count the instances of email spam.  The model has the following
random variables and events:
\begin{itemize}
    \item
        \( [S=1] \), the event that the email is in fact spam, a
        positive outcome occurring with probability \( \Prob{S=1} = \psi
        \),
    \item
        \( [S=0] \), the event that the email is \emph{not} spam, a
        negative outcome with \( \Prob{S = 0} = 1 - \psi \),
    \item
        \( [R=1] \), the event that a spam email is correctly marked
        spam, a \emph{true positive},
    \item
        \( [R=0] \), the event that a nonspam email is \emph{incorrectly}
        marked spam, a \emph{false negative}.
\end{itemize}
Then define the following parameters:
\begin{itemize}
    \item
        \emph{sensitivity}%\index{sensitivity} \( \eta = \Prob{R = 1 \given S =1} = 0.90 \) (so
        the false positive rate is \( \Prob{R = 0 \given S =1} = 0.10 \)),
    \item
        \emph{specificity}%\index{specificity} \( \theta = \Prob{R = 0 \given S = 0} = 0.95 \)
        (so the false negative rate is \( \Prob{R = 1 \given S = 0} =
        0.05 \)),
    \item
        \emph{prevalence}%\index{prevalence} \( \psi = \Prob{S = 1} \).
\end{itemize}
Then
\begin{align*}
    \tau = \Prob{R = 1} &= \Prob{R = 1 \given S =1} \Prob{S = 1} + \Prob
    {R = 1 \given S =0} \Prob{S = 0} \\
    &= \psi \eta + (1 - \psi)(1-\theta).
\end{align*}

Rewrite the equation to solve for the prevalence \( \psi \), usually the
parameter of interest:
\[
    \psi = \frac{\tau + \theta - 1}{\eta + \theta -1}.
\] Assume as an example, that the filter marks \( r = 233 \) emails as
spam out of a total of \( n = 1000 \) emails.  Then the empirical rate
of rejection \( \hat{\tau} = r/n = 0.233 \).  From the relation for the
prevalence \( \psi \) in terms of the other parameters the empirical
estimate of prevalence is \( \hat{\psi} = 0.21529 \).  Since this is a
Bernoulli random variable, the traditional frequentist confidence
interval is \( \tau \pm 1.96 \sqrt{\tau (1-\tau)/n} \) which becomes \(
[0.20680, 0.25920] \).  Using these in the relation for the prevalence \(
\psi \) given an interval \( \psi \in [0.184, 0.246] \).

Unfortunately, if \( r \) is relatively small compared to \( n \), the
relation can give nonsensical estimates.  For example, \( \eta = 0.99 \),
\( \theta = 0.97 \), \( r = 5 \) from \( n = 250 \) gives \( \tau = 0.02
\) but \( \psi = -0.01042 \).

Before starting the Gibbs sampling, first use simple Bayesian inference%
\index{Bayesian inference}
starting from Bayes' Formula%
\index{Bayes' Formula}
\[
    \text{ Posterior} \propto \text{ Prior } \cdot \text{ Likelihood }
\] or in probability notation
\[
    \Prob{\tau \given r} \propto \Prob{\tau} \cdot \Prob{r \given \tau}.
\] Start with a noninformative prior where \( \tau \) is uniformly
distributed on \( (0,1) \), but use a Beta density%
\index{Beta density}
\[
    \Prob{\tau = x} = \frac{1}{B(\alpha, \beta)} x^{\alpha-1}(1-x)^{\beta-1}
\] with initial shape parameters \( \alpha = 1 \) and \( \beta = 1 \) so
the density is uniform.  Recall that \( \Prob{r \given \tau} \) is
binomial, so
\begin{align*}
    \Prob{\tau \given r} &\propto \left( \tau^{\alpha_0 - 1}(1-\tau)^{\beta_0
    - 1} \right) \times \left( \tau^r (1 - \tau)^{n-r} \right) \\
    &\propto \tau^{\alpha_0 + r -1} (1-\tau)^{\beta_0 + n - r -1}.
\end{align*}
The right side is again the form of the Beta distribution. This
correspondence between the Beta families for the prior and the posterior
and the Binomials for the likelihood makes them a \defn{conjugate pair}.%
\index{conjugate pair}%
The mean of the Beta distribution with shape parameters \( \alpha \) and
\( \beta \) is \( \alpha/(\alpha + \beta) \) and the variance is \(
\alpha \beta/((\alpha + \beta)^2(\alpha + \beta + 1)) \).

For the spam classification problem with the previous data
\begin{align*}
    \alpha &= \alpha_0 + r = 1 + 233 = 234 \\
    \beta &= \beta_0 + n -r = 1 + 1000 - 233 = 768.
\end{align*}
This allows using the resulting Beta density posterior for \( p(\tau
\given r) \) to derive the mean and confidence interval for \( \tau \).
Using the facts about the mean and variance above, the posterior density
mean is \( \alpha/(\alpha + \beta) = 0.23353 \).  The \( 95\% \)
confidence interval using the Beta cdf defined by \(
\operatorname{Beta}
(p_{-}, \alpha_0, \beta_0) = 0.025 \) and \(
\operatorname{Beta}
(p_{+}, \alpha_0, \beta_0) = 0.975 \) is \( (0.20786, 0.26021) \).  This
is approximately the same as the frequentist confidence interval.

Next use a Gibbs sampler to get point and interval estimates for \( \psi
\).  Unlike the estimate for \( \tau \) where the count \( r \) of the
spam-marked emails is available, a Gibbs sampler is useful because it is
not feasible to directly count the spam emails.  Again using Bayes'
Formula
\begin{multline*}
    p = \Prob{S=1 \given R = 1} = \Prob{S=1, R = 1}/\Prob{R = 1} = \\
    \Prob{R = 1 \given S=1} \Prob{S = 1}/\Prob{R = 1} = \psi\eta/\tau
\end{multline*}
and
\[
    1 - p = \Prob{S=1 \given R = 0} = \psi(1-\eta)/(1-\tau).
\] Simulate the \emph{latent counts}
\begin{align*}
    X \given (r,\psi) &\sim
    \operatorname{Binom}
    (r, p) \\
    Y \given (r,\psi) &\sim
    \operatorname{Binom}
    (n-r, 1-p) \\
    \psi \given (X,Y) &\sim
    \operatorname{Beta}
    (\alpha_n, \beta_n) \\
\end{align*}
where \( V = X + Y \), \( \alpha_n = \alpha_0 + V \) and \( \beta_n = n
- V \).  Use one value of \( \psi \) to find the next in the typical
Gibbs iteration.  This is in effect a Markov Chain for which the
limiting distribution is the posterior of \( \psi \).  At each
iteration, use the prior distribution and the data.  The continued and
recursive use of this information will cause the convergence of the
simulated values toward the appropriate posterior distribution.  Figure~%
\ref{fig:gibbsampler:spamfilter} illustrates the results of this Gibbs
Sampler method.

\begin{figure}
    \centering
    \includegraphics[scale=0.75]{spamfilter}
    \caption{Results of the Gibbs sampler applied to the spam filter.}%
    \label{fig:gibbsampler:spamfilter}
\end{figure}

\subsection*{Gibbs Sampling in Statistics}

Suppose \( f(x_1, x_2, \dots, x_N) \) is a joint probability density in
which the variables represent parameters of a statistical model.  The
goal is to get point and interval estimates for these parameters.  To
fit this into the Gibbs sampling framework, assume that all the
single-variable conditional probability densities
\[
    f(x_i \given x_j, j \ne i)
\] are available, that is, are a type for which samples can be obtained
using standard algorithms.  Examples of available densities include the
uniform, the normal, the gamma, the Poisson, and any finite
distribution.  To generate a sequence of samples, select \(
\overrightarrow{x}^0 = (x_1^0, x_2^0, \dots, x_N^0) \) arbitrarily and
then create \( \overrightarrow{x}^1 = (x_1^1, x_2^1, \dots, x_N^1) \) as
follows:
\begin{enumerate}
    \item
        Generate a sample \( x_1^1 \) from \( f(x_1 \given x_2^0, x_3^0,
        \dots x_N^0) \).
    \item
        Generate a sample \( x_2^1 \) from \( f(x_2 \given x_1^1, x_3^0,x_4^
        {0} \dots x_N^0) \).
    \item
        Generate a sample \( x_3^1 \) from \( f(x_3 \given x_1^1, x_2^1,
        x_4^0, \dots x_N^0) \).
    \item
        \dots
    \item[N.]
        Generate a sample \( x_N^1 \) from \( f(x_N \given x_1^1, x_2^1,
        \dots x_{N-1}^1) \).
\end{enumerate}
One cycle, similar to a sweep of the lattice in the Ising model or a
raster scan of an image, produces a new value \( \overrightarrow{x}^1 \).
Repeating this process \( M \) times produces
\[
    \overrightarrow{x}^0, \overrightarrow{x}^1, \overrightarrow{x}^2,
    \dots, \overrightarrow{x}^M
\] which approximates a sample from the probability density \( f(x_1, x_2,
\dots, x_N) \).

Using this sample, almost any property of the probability density can be
investigated.  For example, focusing on only the first component of each
\( \overrightarrow{x}^k \) produces a sample
\[
    x_1^0, x_1^1, x_1^2,\dots, x_1^M
\] from the marginal probability distribution of the first component,
formally given by the integral
\[
    f(x_1) = \int_{x_2}\cdots \int_{x_N} f(x_1, x_2, \dots, x_N) \df{x_N}
    \dots \df{x_2}.
\] In this way, Gibbs sampling can be thought of as a multi-dimensional
numerical integration algorithm.  The expected value of the first
component \( x_1 \),
\[
    \E{x_1} = \int_{x_1} x_1 f(x_1) \df{x_1}
\] is estimated by the arithmetic mean of the sample \( x_1^0, x_1^1, x_1^2,\dots,
x_1^M \).  A \( 95\% \) confidence interval for \( x_1 \) can be taken
directly from the sample.

\subsection*{Gibbs Sampling for Normal Parameters}

As another example of the Gibbs sampler, suppose that \( X \sim N(\mu,
1/\tau) \) with \( \mu \) and \( \tau \) unknown.  Based on a reasonably
sized sample, the goal is to get the posterior distributions of \( \mu \)
and \( \tau \) using the Gibbs sampler.  Here \( \mu \) is the
population mean, and \( \tau \), called the \emph{population precision},%\index{population precision} is the
reciprocal of the variance, \( n \) is the sample size, \( \bar{x} \) is
the sample mean and \( s^2 = \frac{1}{n-1} \sum\limits_{i=1}^n (x_i -
\bar{x})^2 \) is the sample variance.  Then make a sequence of
iterations \( i = 1, \dots N \); with a sample \( \mu^{(i)} \) from \( f
(\mu \given \tau^{(i-1)}, \text{data}) \) (see below for the definition)
and sample \( \tau^{(i)} \) from \( f(\tau \given \mu^{(i-1)}, \text{data})
\) (see below for the definition).  Then the theory behind Gibbs
sampling ensures that after a sufficiently many iterations, \( T \), the
set \( \setof{(\mu^{(i)}, \tau^{(i)})}{i = T+1, \dots, N} \) can be seen
as a random sample from the joint posterior distribution. The priors are
that \( f(\mu, \tau) = f_{\mu}(\mu) \times f_{\tau}(\tau) \) with \( f_{\mu}
(\mu) \propto 1 \) and \( f_{\tau}(\tau) \propto 1/\tau \).  With these
definitions and assumptions, standard theory shows the sample mean from
a normal distribution given the population precision has a conditional
posterior distribution
\[
    (\mu \given \tau, \text{data}) \sim N\left(\bar{x}, \frac{1} {n \tau}
    \right).
\]

Next, the claim is that the precision, (the reciprocal of the variance)
given the mean has a conditional posterior distribution
\[
    (\tau \given \mu, \text{data}) \sim
    \operatorname{Gamma}
    \left( \frac{n}{2}, \frac{2}{(n-1) s^2 + n (\mu - \bar{x})^2} \right).
\] The derivation of the claim is:
\[
    \text{ Posterior} \propto \text{ Prior } \times \text{ Likelihood }
\] or in probability notation
\[
    p(\tau \given \mu, \text{data} ) \propto p(\tau) \times p( \mu,
    \text{data} \given \tau).
\] Then using \( \frac{1}{\sigma^n} = \tau^{n/2} \)
\begin{align*}
    p(\tau \given \mu, \text{data} ) &\propto \tau^{-1} \times \tau^{n/2}
    \exp \left( -\frac{\tau}{2} \sum\limits_{\nu=1}^n (x_\nu - \bar{x})^2
    \right) \\
    & \propto \tau^{-1} \times \tau^{n/2} \exp \left( -\frac{\tau}{2}
    \sum\limits_{\nu=1}^n (x_\nu - \mu + \mu - \bar{x})^2 \right) \\
    & \propto \tau^{-1} \times \tau^{n/2} \exp \left( -\frac{\tau}{2}
    \left[ \sum\limits_{\nu=1}^n (x_\nu - \mu)^2 \right.  \right.  \\
    & \qquad \left.  \left.  +2 \sum\limits_{\nu=1}^n (x_\nu - \mu)(\mu
    - \bar{x}) + \sum\limits_{\nu=1}^n (\mu - \bar{x})^2 \right] \right)
    \\
    & \propto \tau^{-1} \times \tau^{n/2} \exp \left( -\frac{\tau}{2}
    \left[ \sum\limits_{\nu=1}^n (x_\nu - \mu)^2 + \sum\limits_{\nu=1}^n
    (\mu - \bar{x})^2 \right] \right) \\
    & \propto \tau^{n/2-1} \times \exp \left( -\frac{\tau}{2} \left[(n-1)
    s^2 + n(\mu - \bar{x})^2 \right] \right).
\end{align*}
The last expression is now recognizable as a Gamma distribution%
\index{Gamma distribution}
with shape parameter \( \frac{n}{2} \) and rate (inverse scale parameter)
\( (n-1) s^2 + n(\mu - \bar{x})^2 \).  Having this distribution is
convenient for implementing the Gibbs sampler.  (As a side remark, in
some derivations for Bayesian statistics, the prior for \( \tau \) is
taken to be \(
\operatorname{Gamma}
(\alpha, \beta) \) which leads to a similar looking Gamma posterior
including the additional parameters \( \alpha \) and \( \beta \), but
for Gibbs sampling, this is unnecessary because the theory shows that
the sampling will converge even with the simpler choice.)

The R script below implements the Gibbs sampler for these parameters,
with \( n = 30 \), a typical value, and postulating \( \bar{x} = 15 \),
and \( s^2 = 3 \).  The number of iterations is \( N = 11{,}000 \) with
a transient period of \( 1000 \) iterations.  Figure~%
\ref{fig:gibbsampler:normalparam} shows the results of the of Gibbs
sampler for the parameters of a normal distribution.

\begin{figure}
    \centering
    \includegraphics[scale=0.75]{normalparam}
    \caption{Results of the Gibbs sampler for the parameters of a normal
    distribution.}%
    \label{fig:gibbsampler:normalparam}
\end{figure}

\subsection*{Bayesian Hierarchical Models}

Apply Gibbs sampling formalism to a \defn{Bayesian hierarchical model}.%
\index{Bayesian hierarchical model}
Hierarchical Bayesian models naturally describe the connections among
data, observed parameters and other unobserved parameters, sometimes
called \defn{latent variables}.%
\index{latent variables}
A simple three-level hierarchical model uses Bayes' rule to bind
together data, \( X \), a parameter to be estimated, \( \lambda \), and
an additional hyper-parameter, \( \beta \).  Both \( \lambda \) and \(
\beta \) can be vectors.  These are connected in the following way:
\begin{enumerate}
    \item
        At the first level, \( X \) is described by its likelihood
        function \( f(X \given \lambda) \), the probability of observing
        \( X \) conditioned on \( \lambda \).
    \item
        At the next level, \( \lambda \) is modeled by a probability
        density function, \( g(\lambda \given \beta) \), conditioned on
        the parameter \( \beta \).
    \item
        At the third level, the hyper-parameter \( \beta \) is modeled
        with another density function \( h(\beta) \).  The choice of \(
        h(\beta) \) reflects the modeler's prior beliefs about the
        likely values of \( \beta \).
\end{enumerate}
The three density functions are combined with Bayes' rule, producing a
probability density function for \( \lambda \) and \( \beta \)
conditioned on the data \( X \)
\[
    F(\lambda, \beta | X) \propto f(X \given \lambda) g(\lambda \given
    \beta) h(\beta).
\] The constant of proportionality is the reciprocal of
\[
    \int_{\lambda} \int_{\beta} f(X \given \lambda) g(\lambda \given
    \beta) h(\beta) \df{\beta} \df{\lambda}
\] which is independent of the parameters \( \lambda \) and \( \beta \),
though dependent on the data \( X \).  The integrals, (or sums, in the
case of discrete distributions) are over all values \( \lambda \) and \(
\beta \).  In most cases, the integral or sum is impossible to evaluate.
However, as before, the Metropolis-Hastings algorithm avoids this
expression.

As a specific example, consider a model of water pump failure rates. The
data, \( X \), are given by pairs \( (s_i, t_i) \) for \( i = 1,2, \dots,
10 \).  Each pair represents failure information for an individual pump.
For each pump, assume the number of failures \( s_i \) in time \( t_i \)
is given by a Poisson distribution with parameter \( \lambda_i t_i \),
that is
\[
    f_i(s_i \given \lambda_i) = \frac{(\lambda_i t_i)^{s_i} \EulerE^{-\lambda_i
    t_i}}{s_i!}, \quad i = 1,2, \dots, 10.
\]

Assuming the failures occur independently, the likelihood function for \(
\overrightarrow{\lambda} = (\lambda_1, \lambda_2, \dots, \lambda_{10}) \)
is the product:
\[
    f(X \given \overrightarrow{\lambda}) = \prod\limits_{i=1}^{10} \frac
    {(\lambda_i t_i)^{s_i} \EulerE^{-\lambda_i t_i}}{s_i!}.
\] The traditional frequentist approach is to use \( \bar{\lambda_i} = s_i/t_i
\) as the point estimate of \( \lambda_i \) for \( i = 1,2, \dots, 10 \).
The Bayesian approach is to assume that the individual \( \lambda_i \)'s
are linked together by a common distribution.  A natural choice is a
gamma distribution with shape parameter \( \alpha \) and scale parameter
\( \beta \), so that the density for the \( i \)th parameter is
\[
    g_i(\lambda_i \given \alpha, \beta) = \frac{\lambda_i^ {\alpha-1}
    \EulerE^{-\lambda_i/\beta}}{\beta^{\alpha} \Gamma(\alpha)}.
\] The remaining hyper-parameter \( \beta \) is described by an inverse
gamma distribution%
\index{inverse gamma distribution}
with shape parameter \( \gamma \) and scale parameter \( \delta \), so
that
\[
    h(\beta) = \frac{\delta^{\gamma} \EulerE^{-\delta/\beta}}{\beta^{\gamma+1}
    \Gamma(\gamma)}.
\] The parameters \( \gamma \) and \( \delta \) are selected to make the
top-level inverse gamma reasonably \emph{diffuse}.  A diffuse
distribution%
\index{diffuse distribution}
tries to convey as little prior information as possible about the
parameters.  An extreme case of a noninformative distribution is the
uniform distribution on the parameter space.

The resulting posterior joint density for the parameters \( \lambda_1,
\lambda_2, \dots, \lambda_{10} \) along with the scale parameter \(
\beta \) is
\begin{multline*}
    F(\lambda_1, \lambda_2, \dots, \lambda_{10}, \beta \given X) \propto
    \\
    \left[ \prod_{i=1}^{10} \frac{(\lambda_i t_i)^{s_i} \EulerE^{-\lambda_i
    t_i}}{s_i!} \right] \left[ \prod_{i=1}^{10} \frac{\lambda_i^{\alpha-1}
    \EulerE^{-\lambda_i/\beta}}{\beta^{\alpha} \Gamma(\alpha)} \right]
    \left[ \frac{\delta^{\gamma} \EulerE^{-\delta/\beta}}{\beta^{\gamma+1}
    \Gamma(\gamma)} \right].
\end{multline*}
For \( i = 1,2, \dots, 10 \), the density for \( \lambda_i \)
conditioned on the other parameters is proportional to
\[
    \lambda_i^{s_i + \alpha -1} \EulerE^{-\lambda_i (t_i+1/\beta)}.
\] The constant of proportionality is obtained by absorbing all factors
independent of \( \lambda_i \).  The form of the density for \( \lambda_i
\) shows that \( \Prob{\lambda_i \given \lambda_j, j \ne i, X, \beta} \)
is a gamma distribution with shape parameter \( s_i + \alpha \) and
scale parameter \( 1/(t_i + 1/\beta) \).  Since the gamma distribution
is available, Gibbs sampling can be applied at this step.  The density
for \( \beta \), conditioned on the other parameters, is proportional to
\[
    \frac{\EulerE^{-({\sum_{i=1}^{10}\lambda_i + \delta})/\beta}}{\beta^
    {10\alpha+\gamma+1}}
\] showing that \( \Prob{\beta \given \lambda_1, \lambda_2, \dots,
\lambda_{10}, X} \) is an inverse gamma distribution with shape
parameter \( \gamma + 10 \alpha \) and \( \sum\limits_{i=1}^{10} \lambda_i+\delta
\). This too is an available distribution.

This model is an example of a \emph{conjugate hierarchical model}.%\index{conjugate hierarchical model}
In a conjugate hierarchical model the intermediate distributions, in
this case for \( \lambda_i \) and \( \beta \), are similar to the
original distributions in the hierarchical model.  This fits with the
Gibbs sampling requirement that these distributions be available.

\subsection*{Gibbs Sampling for Digital Images}

A simple model of a digital image consists of pixel elements arranged on
a rectangular lattice with \( N \) sites.  Each pixel takes a value from
a set \( S = \set{1,2, \dots, K} \), indicating grayscale or color.  For
simplicity and analogy to the Ising model, take a black and white image
with pixel levels \( -1, +1 \). An image is a configuration \( \omega
\in \Omega \) assigning a level to each of the \( N \) sites.  Even
modestly sized images result in immensely large configuration spaces.
For a \( 100 \times 100 \) black and white image, \( \card{\Omega} = 2^{10
{,}000} \).

Consider a model for image degradation with additive noise, modeled by \(
N \) independent identically distributed random variables \( \mathcal{N}
= \set{\eta_1, \eta_2, \dots \eta_N} \).  Specifically, take noise with
the \( \eta_i \) normally distributed with mean \( 0 \) and variance \(
\sigma^2 \), that is \( \eta_i \sim N(0, \sigma^2) \). Letting \( \omega^
{\text{blurred}} \) indicate the degraded or blurred image, \( \omega^{\text
{blurred}} = \omega + \mathcal{N} \).  Since the values of \( \omega^{\text
{blurred}} \) are real numbers, the resulting image is determined by
rounding each value to the nearest value in \( S \).

The relationship between the original image and the degraded version is
probabilistic, given any image \( \omega \), there is some probability a
particular \( \omega^{\text{blurred}} \) is the degraded version of \(
\omega \).  Image reconstruction looks at the problem the other way
around; given \( \omega^{\text{blurred}} \), there is some probability \(
\omega \) is the original image.  This leads to an application of Bayes'
Rule.  The \emph{posterior distribution}%
\index{posterior distribution}
for \( \omega \) conditioned on \( \omega^{\text{blurred}} \) is
\[
    \Prob{\omega \given \omega^{\text{blurred}}} = \frac{\Prob{\omega^{\text
    {blurred}} \given \omega} \Prob{\omega}} {\Prob{\omega^{\text{blurred}}}}.
\] The goal is to find the configuration maximizing \( \Prob{\omega
\given \omega^{\text{blurred}}} \), called the \defn{maximum a
posteriori estimate}.%
\index{maximum a
  posteriori estimate}
The technique is to formulate a new version of the Metropolis algorithm
with Gibbs sampling.

By the Law of Total Probability the denominator is
\[
    \Prob{ \omega^{\text{blurred}}} = \int\limits_{\omega \in \Omega}
    \Prob{\omega^{\text{blurred}} \given \omega} \Prob{\omega} \df{\omega}.
\] This integral (or sum) is over all \( \omega \in \Omega \) and does
not depend on \( \omega \).  This is reminiscent of the partition
function and recalling the Metropolis algorithm just ignore it.

The likelihood function \( \Prob{\omega^{\text{blurred}} \given \omega} \)
is%
\index{likelihood function}
\[
    \Prob{\omega^{\text{blurred}} \given \omega} \propto \prod_{i=1}^N
    \EulerE^{-\frac{ (\omega_i^{\text{blurred}} - \omega_i)^2}{2 \sigma^2}}
    = \EulerE^{ -\frac{1}{2\sigma^2} \sum\limits_{i=1}^N (\omega_i^{\text
    {blurred}} - \omega_i)^2}
\] where any constant of proportionality will be absorbed into the
denominator above.

An image has patterns which are contiguous regions of similar pixel
values.  This is reminiscent of the Ising model, where after
magnetization the lattice has long-range correlation between pixels,
that is, image-like features. On the other hand, if neighboring values
are uncorrelated, the result is the visual equivalent of noise, also
reminiscent of the Ising model at high temperatures.

This suggests using the Boltzmann probability distribution with the
Ising potential energy function as the prior distribution on images,
that is
\[
    \Prob{\omega} \propto \EulerE^{-E_{\text{ising}}(\omega)/({\text k}_
    {\text{Boltz}}T)}
\] where
\[
    E_{\text{ising}}(\omega) = -J \sum\limits_{i=1}^{N} \sum\limits_{k=1}^4
    \omega_i \omega_{\left\langle i \right\rangle[k]}
\] is the nearest neighbor affinity.  To keep the idea of correlated
pixel values, let \( {\text k}_{\text{Boltz}}T/J = 1 < T_c \), the
critical temperature below which a phase transition occurs.

Putting all the parts together, the posterior distribution, \( \Prob{\omega
\given \omega^{\text{blurred}}} \) is
\begin{align*}
    \Prob{\omega \given \omega^{\text{blurred}}} &\propto \EulerE^{-\frac
    {1}{2\sigma^2 } \sum\limits_{i=1}^N(\omega_i^{\text{blurred}} -
    \omega_i)^{2}} \EulerE^{-E_{\text{ising}}} \\
    &\propto \EulerE^{-\left[ \frac{1}{2\sigma^2 } \sum\limits_{i=1}^N(\omega_i^
    {\text{blurred}} - \omega_i)^{2} + E_{\text{ising}} \right] }.
\end{align*}
Viewing this from a statistical mechanics perspective leads to an analog
of an energy function
\begin{align*}
    E_{\text{image}}(\omega \given \omega^{\text{blurred}}) &= \frac{1}{2\sigma^2
    } \sum\limits_{i=1}^N(\omega_i^{\text{blurred}} - \omega_i)^{2} + E_
    {\text{ising}} \\
    &= \frac{1}{2\sigma^2 } \sum\limits_{i=1}^N(\omega_i^{\text{blurred}}
    - \omega_i)^{2} - \sum\limits_{\langle i\rangle [j]} \omega_i\omega_j.
\end{align*}
where \( {\langle i \rangle} \) indicates the set of nearest neighbor
sites \( j \) for site \( i \).

Finding the most probable original image \( \omega \) given \( \omega^{\text
{blurred}} \) is equivalent to minimizing \( E_{\text{image}}(\omega
\given \omega^{\text{blurred}}) \).  The first term is a positive
potential energy penalty for straying too far from the data \( \omega^{\text
{blurred}} \) while the second term represents a negative potential
energy reflecting the desire to align neighboring pixel values.  The
optimal solution balances these two conflicting objectives.

Creating an energy function to minimize leads to a contrast in methods.
The Ising model starts with an objective function and interprets it as a
energy function, using this to convert it to a probability from a
physical interpretation.  The image reconstruction model starts with a
probabilistic situation with a Bayesian structure, leading to an energy
function.

To implement Gibbs sampling, the probability of \( \omega_i \)
conditioned on \emph{all} the other sites depends on \emph{only} the
sites in the nearest neighborhood set.  Suppressing the dependence on \(
\omega^{\text {blurred}} \), this means
\begin{align*}
    \Prob{\omega_i \given \omega_j, j \ne i } &= \Prob { \omega_i \given
    \omega_j, j \in \langle i \rangle } \\
    & \propto \EulerE^{-E_i(\omega_i \given \omega_j, j \in \langle i
    \rangle)}
\end{align*}
where
\[
    E_i(\omega_i \given \omega_j, j \in \langle i\rangle) = \frac{1}{2\sigma^2}
    (\omega_i^{\text{blurred}}-\omega_i)^2 - \sum\limits_{\langle i
    \rangle[j]} \omega_i\omega_j.
\] A probability distribution whose conditional probabilities depend on
only the values in a neighborhood system is called a \defn{Gibbs
distribution}%
\index{Gibbs distribution}
and is an example of a larger notion called a \emph{Markov random field}.%
\index{Markov random field}

A standard way to implement Gibbs sampling for images is to use a
sequence of raster scans, in order by rows or columns, guaranteeing all
sites are visited many times.  At a selected site \( i \), select \(
\omega_i = k \) with probability
\[
    \Prob{\omega_i = \ell} \propto \EulerE^{-\frac{1}{2\sigma^2}(\omega_i^
    {\text {blurred}}-\ell)^2 - \sum\limits_{\langle i \rangle [j]} \ell
    \omega_j}.
\] Repeating this with many raster scans results in a sequence of images
that approximates a sample from the posterior distribution.  Note the
connection to the previous simple Gibbs sampling algorithm, generating a
sample from \( f(x) \) by sampling iteratively from the conditional
distributions, except that now there are many more conditional
distributions, one for each pixel.

Gibbs sampling fits into the Hastings generalization of the Metropolis
algorithm in the following way:  In the second step of the
Metropolis-Hastings algorithm, the probabilities \( \alpha_{ij} \) are
all equal to \( 1 \).  However, the transitions are no longer
time-independent, since each depends on the site choice.  As a result,
the proof is somewhat more involved than the original proofs of
convergence given by Hastings.  Gibbs sampling will produce a sequence
representing a sample from \( \Prob{\omega \given \omega^{\text{blurred}}}
\).  The full algorithm also includes a ``temperature'' parameter \( T \)
to create a simple form of simulated annealing.

\begin{theorem}
    Assume
    \begin{enumerate}
        \item
            an image with \( N \) pixels,
        \item
            \( T_k \) is a any decreasing sequence of temperatures such
            that
            \begin{enumerate}
                \item
                    \( T_k \to 0 \) as \( k \to \infty \),
                \item
                    \( T_k \ge N \Delta/\ln(k) \) for all sufficiently
                    large \( k \) and constant \( \Delta \).
            \end{enumerate}
    \end{enumerate}
    Then starting at \( \omega^{(0)} = \omega^{\text{blurred}} \), the
    Gibbs sampling sequence \( \omega^{(k)} \) for \( k=0,1,2, \dots \)
    converges in distribution to the distribution which is uniform on
    the minimum vales of \( E_{\text{image}}(\omega) \) and \( 0 \)
    otherwise.
\end{theorem}

In other words, following a prescribed annealing schedule, Gibbs
sampling must, in theory, produce a maximum a posteriori estimate of \(
\Prob{\omega \given \omega^{\text{blurred}}} \).

Even though this result guarantees convergence to the most likely image,
the rate of convergence is slow.  Theoretically, for a \( 100 \times 100
\) lattice with \( N = 10^4 \) pixels, using the theorem requires \(
\EulerE^{20{,}000} \) steps to go from \( T = 4 \) to \( T = 0.5 \).  In
practice, it takes about \( 300 - 1000 \) raster scans to produce
acceptable results.

For a black and white (\( k = -1 \) or \( 1 \)) image Gibbs sampling
with annealing is especially straightforward to implement using the
ideas of the section on Gibbs sampling.  At the pixel \( \omega_i \)
define
\[
    E^k = \frac{1}{2 \sigma^2}( k - \omega_i^{\text{blurred}})^2 + k
    \sum\limits_{\langle i \rangle[j]} \omega_j.
\] Set \( \omega_i = k \) with probability
\[
    \frac{\EulerE^{-E^k/{\text k}_{\text{Boltz}}T}}{\EulerE^{-E^{-1}/({\text
    k}_{\text{Boltz}}T)} + \EulerE^{-E^1/({\text k}_{\text{Boltz}}T)}}.
\]

\visual{Section Starter Question}{../../../../CommonInformation/Lessons/question_mark.png}
\section*{Section Ending Answer}

For two continuous random variables \( X \) and \( Y \), Bayes' rule can
be stated in terms of the conditional and marginal densities,
\begin{align*}
    f_{X \given Y = y}(x) &= \frac{f_{X,Y}(x,y)}{f_Y(y)}, \\
    f_{Y \given X = x}(y) &= \frac{f_{X,Y}(x,y)}{f_X(x)}.
\end{align*}
Therefore
\[
    f_{X \given Y=y}(x) = \frac{f_{Y \given X = x}(y) f_X(x)}{f_Y(y)}.
\]

\subsection*{Sources} The sections on the general theory of the Gibbs
sampler and the bivariate binomial are adapted from the article
``Explaining the Gibbs Sampler'' by George Casella and Edward George!
\cite{casella92}.  The section on Gibbs Sampling from the Bivariate
Normal is adapted from ``A simple Gibbs sampler'' by Darren Wilkinson
\cite{wilkinson}.  The subsection on A Gibbs Sampler for a Spam Filter
is adapted from ``Three Simple Applications of Markov Chains and Gibbs
Sampling'' by Gui Larangeira
\cite{larangeira}.  The subsection on theory and Bayesian hierarchical
models is adapted from ``The Evolution of Markov Chain Monte Carlo
Methods'' by Richey
\cite{richey10}.

\hr

\visual{Algorithms, Scripts, Simulations}{../../../../CommonInformation/Lessons/computer.png}
\section*{Algorithms, Scripts, Simulations}

\subsection*{Algorithm}

\begin{algorithm}
  \DontPrintSemicolon
  \KwData{Number of iterations \( N \), ``Burn in'' iterations \( T \), distributions
    and parameters}
  \KwResult{Plots of Gibbs sampler results}
  Initialize \( X_0, Y_0 \)\;
  \For{\( 1:N \)}{
    Generate \( X'_j \sim f(x \given Y'_j = y'_j) \)\;
    Generate \( Y'_{j+1} \sim f(y \given X'_j = x'_j) \)\;
  }
  Plot images of resulting distributions and parameters from \( T \)
  to \( N \);
  \caption{Gibbs sampler algorithm}
\end{algorithm}

\subsection*{Scripts}

\begin{description}

% \item[Geogebra] 

% \link{  .ggb}{GeoGebra applet}

\item[R] 

\link{http://www.math.unl.edu/~sdunbar1/bivariatenormalsampler.R}{R script for
Gibbs sampler for marginals from bivariate normal.}

\begin{lstlisting}[language=R]
gibbs<-function (n, rho) 
{
        mat <- matrix(ncol = 2, nrow = n)
        x <- 0
        y <- 0
        mat[1, ] <- c(x, y)
        for (i in 2:n) {
                x <- rnorm(1, rho * y, sqrt(1 - rho^2))
                y <- rnorm(1, rho * x, sqrt(1 - rho^2))
                mat[i, ] <- c(x, y)
        }
        mat
}

f  <- function(x) { return( (1/sqrt(2 * pi)) * exp(-x^2/2) )}

N  <-  1000
rho  <-  0.8

bvngs <- gibbs(N, rho)
bvn <- bvngs[(N/2 + 1):N, ]

par(mfrow=c(4,2))
hist(bvn[,1], freq=FALSE, 40)
curve(f, -4, 4, add=TRUE, col="red")
hist(bvn[,2], freq=FALSE, 40)
curve(f, -4, 4, add=TRUE, col="red")
qqnorm(bvn[ , 1], main="Q-Q Plot of bvn[, 1]")
qqnorm(bvn[ , 2], main="Q-Q Plot of bvn[ ,2]")
plot(bvn, col=1:500)
plot(bvn, type="l")
plot(ts(bvn[ , 1]))
plot(ts(bvn[ , 2]))
par(mfrow=c(1,1))
\end{lstlisting}
\item[R] 

\link{http://www.math.unl.edu/~sdunbar1/normalparam.R}{R script for
Gibbs sampler for normal parameters.}

\begin{lstlisting}[language=R]
# summary statistics of sample
n    <- 30
ybar <- 15
s2   <- 3

# sample from the joint posterior (mu, tau | data)
mu      <- rep(NA, 11000)
tau <- rep(NA, 11000)
T       <- 1000    # burnin
tau[1] <-  1
## tau[1]  <- 1  # initialisation
for(i in 2:11000) {   
    mu[i]  <- rnorm(n = 1, mean = ybar, sd = sqrt(1 / (n * tau[i - 1])))
##    sigmasq[i] <- sigmasq[i-1] * rchisq(n = 1, n-1) / (n - 1)
    tau[i] <- rgamma(n = 1, shape = n / 2, scale = 2 / ((n - 1) * s2 + n * (mu[i] - ybar)^2))
}
mu  <- mu[-(1:T)]   # remove burnin
tau <- tau[-(1:T)] # remove burnin

par(mfrow=c(1,2))
hist(mu)
hist(tau)
par(mfrow=c(1,1))
\end{lstlisting}  

\link{http://www.math.unl.edu/~sdunbar1/bivariatenormalsampler.R}{R script for
Gibbs sampler for estimating spam prevalence.}

\begin{lstlisting}[language=R]
N <- 10000            # Number of iterations
Nb <- 2000; N1 <- N+1  # Burn-in

psi <- numeric(N) 
psi[1] <- .5          # Initial value

alpha.0  <- 1.0
beta.0  <- 1.0
eta <- 0.99
theta <- 0.97
r <- 233
n  <- 1000

for(i in 2:N) {      # Gibbs Sampler Loop
tau<-psi[i-1]*eta+(1-psi[i-1])* (1-theta)        
X <- rbinom(1, r, psi[i-1]* eta/tau)
Y <- rbinom(1, n-r, psi[i-1]*(1-eta)/(1-tau))
psi[i] <- rbeta(1, alpha.0+X+Y, beta.0+n-X-Y)
}

gspsi  <- mean(psi[Nb:N])

par(mfrow=c(2,2))
hist(psi)
plot(1:N,cumsum(psi)/(1:N),type="l",ylab= "psi", ylim=c(0.20,0.24))
plot(psi,type='p',pch='.',ylim=c(0.15,0.30))
acf(psi)
\end{lstlisting}

% \item[Octave]

% \link{http://www.math.unl.edu/~sdunbar1/    .m}{Octave script for .}

% \begin{lstlisting}[language=Octave]

% \end{lstlisting}

% \item[Perl] 

% \link{http://www.math.unl.edu/~sdunbar1/    .pl}{Perl PDL script for .}

% \begin{lstlisting}[language=Perl]

% \end{lstlisting}

% \item[SciPy] 

% \link{http://www.math.unl.edu/~sdunbar1/    .py}{Scientific Python script for .}

% \begin{lstlisting}[language=Python]

% \end{lstlisting}

\end{description}



%%% Local Variables:
%%% mode: latex
%%% TeX-master: t
%%% End:


\hr

\visual{Problems to Work}{../../../../CommonInformation/Lessons/solveproblems.png}
\section*{Problems to Work for Understanding}

\renewcommand{\theexerciseseries}{}
\renewcommand{\theexercise}{\arabic{exercise}}

\begin{exercise}
    The probability density function for the bivariate normal
    distribution with means \( 0 \) and variance \( 1 \) for the
    variables and correlation \( \rho \) between the two variables is
    \[
        f(x, y) = \frac{1}{2\pi \sqrt{1 - \rho^2}} \EulerE^{- \frac{x^2
        - 2 \rho x y + y^2}{2(1-\rho^2)}}.
    \] Show that the \( x \) marginal is
    \[
        f_{x}(x) = \frac{1}{\sqrt{2\pi}} \EulerE^{-x^2/2}.
    \]
\end{exercise}
\begin{solution}
    \begin{align*}
        &\int\limits_{-\infty}^{\infty}f(x, y) \df{y} = \int\limits_{-\infty}^
        {\infty}\frac{1}{2\pi \sqrt{1 - \rho^2}} \EulerE^{- \frac{x^2 -
        2 \rho x y + y^2}{2(1-\rho^2)}} \df{y} \\
        &\qquad= \frac{1}{2\pi \sqrt{1 - \rho^2}} \EulerE^{- \frac{x^2}{2
        (1-\rho^2)}} \EulerE^{\frac{\rho^2 x^2}{2(1-\rho^2)}} \int\limits_
        {-\infty}^{\infty} \EulerE^{- \frac{\rho^2 x^2- 2 \rho x y + y^2}
        {2(1-\rho^2)}} \df {y} \\
        &\qquad= \frac{1}{2\pi \sqrt{1 - \rho^2}} \EulerE^{- \frac{x^2}{2
        (1-\rho^2)}} \EulerE^{\frac{\rho^2 x^2}{2(1-\rho^2)}} \int\limits_
        {-\infty}^{\infty} \EulerE^{- \frac{(\rho x - y)^2}{2(1-\rho^2)}}
        \df{y} \\
        &\qquad= \frac{1}{2\pi \sqrt{1 - \rho^2}} \EulerE^{- \frac{x^2}{2
        (1-\rho^2)}} \EulerE^{\frac{\rho^2 x^2}{2(1-\rho^2)}} \int\limits_
        {-\infty}^{\infty} \EulerE^{- \frac{z^2}{2(1-\rho^2)}} \df{z} \\
        &\qquad= \frac{1}{2\pi \sqrt{1 - \rho^2}} \EulerE^{- \frac{x^2}{2
        (1-\rho^2)}} \EulerE^{\frac{\rho^2 x^2}{2(1-\rho^2)}} \int\limits_
        {-\infty}^{\infty} \EulerE^{- \frac{u^2}{2}} \sqrt{1-\rho^2}\df{u}
        \\
        &\qquad= \frac{1}{2\pi \sqrt{1 - \rho^2}} \EulerE^{- \frac{x^2}{2
        (1-\rho^2)}} \EulerE^{\frac{\rho^2 x^2}{2(1-\rho^2)}} \sqrt{2\pi
        (1 - \rho^2)} \\
        &\qquad= \frac{1}{\sqrt{2\pi}} \EulerE^{- \frac{x^2}{2}}.
    \end{align*}
\end{solution}
\begin{exercise}
    The probability density function for the bivariate normal
    distribution with means \( 0 \) and variance \( 1 \) for the
    variables and correlation \( \rho \) between the two variables is
    \[
        f(x, y) = \frac{1}{2\pi \sqrt{1 - \rho^2}} \EulerE^{- \frac{x^2
        - 2 \rho x y + y^2}{2(1-\rho^2)}}.
    \] Show that the \( x \) conditional given \( y \) is
    \[
        f_{x \given y}(x \given y) = \frac{1}{\sqrt{2\pi (1 - \rho^2)}}
        \EulerE^{- \frac{(x - \rho y^)2}{2(1-\rho^2)} }.
    \]
\end{exercise}
\begin{solution}
    \begin{align*}
        &f_{x \given y}(x \given y) = f(x,y)/f_y(y) \\
        &\qquad= \frac{1}{2\pi \sqrt{1 - \rho^2}} \EulerE^{- \frac{x^2 -
        2 \rho x y + y^2}{2(1-\rho^2)}} \left / \right. \frac{1}{\sqrt{2\pi}}
        \EulerE^{-y^2/2} \\
        &\qquad= \frac{1}{\sqrt{2\pi (1 - \rho^2)}} \EulerE^{- \frac{x^2
        - 2 \rho x y + y^2}{2(1-\rho^2)} + \frac{(1-\rho^2) y^2}{2(1-\rho^2)}}
        \\
        &\qquad= \frac{1}{\sqrt{2\pi (1 - \rho^2)}} \EulerE^{- \frac{x^2
        - 2 \rho x y + \rho^2 y^2}{2(1-\rho^2)} } \\
        &\qquad= \frac{1}{\sqrt{2\pi (1 - \rho^2)}} \EulerE^{- \frac{(x
        - \rho y)^2}{2(1-\rho^2)} }.
    \end{align*}
\end{solution}

\hr

\visual{Books}{../../../../CommonInformation/Lessons/books.png}
\section*{Reading Suggestion:}

\bibliography{../../../../CommonInformation/bibliography}

%   \begin{enumerate}
%     \item
%     \item
%     \item
%   \end{enumerate}

\hr

\visual{Links}{../../../../CommonInformation/Lessons/chainlink.png}
\section*{Outside Readings and Links:}
\begin{enumerate}
    \item
    \item
    \item
    \item
\end{enumerate}

\section*{\solutionsname} \loadSolutions

\hr

\mydisclaim \myfooter

Last modified:  \flastmod

\end{document}

%%% Local Variables:
%%% TeX-master: t
%%% End:
