\begin{description}

% \item[Geogebra] 

% \link{  .ggb}{GeoGebra applet}

\item[R] 

\link{http://www.math.unl.edu/~sdunbar1/bivariatenormalsampler.R}{R script for
Gibbs sampler for marginals from bivariate normal.}

\begin{lstlisting}[language=R]
gibbs<-function (n, rho) 
{
        mat <- matrix(ncol = 2, nrow = n)
        x <- 0
        y <- 0
        mat[1, ] <- c(x, y)
        for (i in 2:n) {
                x <- rnorm(1, rho * y, sqrt(1 - rho^2))
                y <- rnorm(1, rho * x, sqrt(1 - rho^2))
                mat[i, ] <- c(x, y)
        }
        mat
}

f  <- function(x) { return( (1/sqrt(2 * pi)) * exp(-x^2/2) )}

N  <-  1000
rho  <-  0.8

bvngs <- gibbs(N, rho)
bvn <- bvngs[(N/2 + 1):N, ]

par(mfrow=c(4,2))
hist(bvn[,1], freq=FALSE, 40)
curve(f, -4, 4, add=TRUE, col="red")
hist(bvn[,2], freq=FALSE, 40)
curve(f, -4, 4, add=TRUE, col="red")
qqnorm(bvn[ , 1], main="Q-Q Plot of bvn[, 1]")
qqnorm(bvn[ , 2], main="Q-Q Plot of bvn[ ,2]")
plot(bvn, col=1:500)
plot(bvn, type="l")
plot(ts(bvn[ , 1]))
plot(ts(bvn[ , 2]))
par(mfrow=c(1,1))
\end{lstlisting}
\item[R] 

\link{http://www.math.unl.edu/~sdunbar1/normalparam.R}{R script for
Gibbs sampler for normal parameters.}

\begin{lstlisting}[language=R]
# summary statistics of sample
n    <- 30
ybar <- 15
s2   <- 3

# sample from the joint posterior (mu, tau | data)
mu      <- rep(NA, 11000)
tau <- rep(NA, 11000)
T       <- 1000    # burnin
tau[1] <-  1
## tau[1]  <- 1  # initialisation
for(i in 2:11000) {   
    mu[i]  <- rnorm(n = 1, mean = ybar, sd = sqrt(1 / (n * tau[i - 1])))
##    sigmasq[i] <- sigmasq[i-1] * rchisq(n = 1, n-1) / (n - 1)
    tau[i] <- rgamma(n = 1, shape = n / 2, scale = 2 / ((n - 1) * s2 + n * (mu[i] - ybar)^2))
}
mu  <- mu[-(1:T)]   # remove burnin
tau <- tau[-(1:T)] # remove burnin

par(mfrow=c(1,2))
hist(mu)
hist(tau)
par(mfrow=c(1,1))
\end{lstlisting}  

\link{http://www.math.unl.edu/~sdunbar1/bivariatenormalsampler.R}{R script for
Gibbs sampler for estimating spam prevalence.}

\begin{lstlisting}[language=R]
N <- 10000            # Number of iterations
Nb <- 2000; N1 <- N+1  # Burn-in

psi <- numeric(N) 
psi[1] <- .5          # Initial value

alpha.0  <- 1.0
beta.0  <- 1.0
eta <- 0.99
theta <- 0.97
r <- 233
n  <- 1000

for(i in 2:N) {      # Gibbs Sampler Loop
tau<-psi[i-1]*eta+(1-psi[i-1])* (1-theta)        
X <- rbinom(1, r, psi[i-1]* eta/tau)
Y <- rbinom(1, n-r, psi[i-1]*(1-eta)/(1-tau))
psi[i] <- rbeta(1, alpha.0+X+Y, beta.0+n-X-Y)
}

gspsi  <- mean(psi[Nb:N])

par(mfrow=c(2,2))
hist(psi)
plot(1:N,cumsum(psi)/(1:N),type="l",ylab= "psi", ylim=c(0.20,0.24))
plot(psi,type='p',pch='.',ylim=c(0.15,0.30))
acf(psi)
\end{lstlisting}

% \item[Octave]

% \link{http://www.math.unl.edu/~sdunbar1/    .m}{Octave script for .}

% \begin{lstlisting}[language=Octave]

% \end{lstlisting}

% \item[Perl] 

% \link{http://www.math.unl.edu/~sdunbar1/    .pl}{Perl PDL script for .}

% \begin{lstlisting}[language=Perl]

% \end{lstlisting}

% \item[SciPy] 

% \link{http://www.math.unl.edu/~sdunbar1/    .py}{Scientific Python script for .}

% \begin{lstlisting}[language=Python]

% \end{lstlisting}

\end{description}



%%% Local Variables:
%%% mode: latex
%%% TeX-master: t
%%% End:
